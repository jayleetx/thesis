% This is the Reed College LaTeX thesis template. Most of the work
% for the document class was done by Sam Noble (SN), as well as this
% template. Later comments etc. by Ben Salzberg (BTS). Additional
% restructuring and APA support by Jess Youngberg (JY).
% Your comments and suggestions are more than welcome; please email
% them to cus@reed.edu
%
% See http://web.reed.edu/cis/help/latex.html for help. There are a
% great bunch of help pages there, with notes on
% getting started, bibtex, etc. Go there and read it if you're not
% already familiar with LaTeX.
%
% Any line that starts with a percent symbol is a comment.
% They won't show up in the document, and are useful for notes
% to yourself and explaining commands.
% Commenting also removes a line from the document;
% very handy for troubleshooting problems. -BTS

% As far as I know, this follows the requirements laid out in
% the 2002-2003 Senior Handbook. Ask a librarian to check the
% document before binding. -SN

%%
%% Preamble
%%
% \documentclass{<something>} must begin each LaTeX document
\documentclass[12pt,twoside]{reedthesis}
% Packages are extensions to the basic LaTeX functions. Whatever you
% want to typeset, there is probably a package out there for it.
% Chemistry (chemtex), screenplays, you name it.
% Check out CTAN to see: http://www.ctan.org/
%%
\usepackage{graphicx,latexsym}
\usepackage{amsmath}
\usepackage{amssymb,amsthm}
\usepackage{longtable,booktabs,setspace}
\usepackage{chemarr} %% Useful for one reaction arrow, useless if you're not a chem major
\usepackage[hyphens]{url}
% Added by CII
\usepackage{hyperref}
\usepackage{lmodern}
\usepackage{float}
\floatplacement{figure}{H}
% End of CII addition
\usepackage{rotating}

% Next line commented out by CII
%%% \usepackage{natbib}
% Comment out the natbib line above and uncomment the following two lines to use the new
% biblatex-chicago style, for Chicago A. Also make some changes at the end where the
% bibliography is included.
%\usepackage{biblatex-chicago}
%\bibliography{thesis}


% Added by CII (Thanks, Hadley!)
% Use ref for internal links
\renewcommand{\hyperref}[2][???]{\autoref{#1}}
\def\chapterautorefname{Chapter}
\def\sectionautorefname{Section}
\def\subsectionautorefname{Subsection}
% End of CII addition

% Added by CII
\usepackage{caption}
\captionsetup{width=5in}
% End of CII addition

% \usepackage{times} % other fonts are available like times, bookman, charter, palatino

% Syntax highlighting #22

% To pass between YAML and LaTeX the dollar signs are added by CII
\title{My Final College Paper}
\author{Jay Lee}
% The month and year that you submit your FINAL draft TO THE LIBRARY (May or December)
\date{May 2019}
\division{Mathematics and Natural Sciences}
\advisor{Heather Kitada Smalley}
\institution{Reed College}
\degree{Bachelor of Arts}
%If you have two advisors for some reason, you can use the following
% Uncommented out by CII
% End of CII addition

%%% Remember to use the correct department!
\department{Mathematics - Statistics}
% if you're writing a thesis in an interdisciplinary major,
% uncomment the line below and change the text as appropriate.
% check the Senior Handbook if unsure.
%\thedivisionof{The Established Interdisciplinary Committee for}
% if you want the approval page to say "Approved for the Committee",
% uncomment the next line
%\approvedforthe{Committee}

% Added by CII
%%% Copied from knitr
%% maxwidth is the original width if it's less than linewidth
%% otherwise use linewidth (to make sure the graphics do not exceed the margin)
\makeatletter
\def\maxwidth{ %
  \ifdim\Gin@nat@width>\linewidth
    \linewidth
  \else
    \Gin@nat@width
  \fi
}
\makeatother

\renewcommand{\contentsname}{Table of Contents}
% End of CII addition

\setlength{\parskip}{0pt}

% Added by CII

\providecommand{\tightlist}{%
  \setlength{\itemsep}{0pt}\setlength{\parskip}{0pt}}

\Acknowledgements{
I want to thank a few people.
}

\Dedication{
You can have a dedication here if you wish.
}

\Preface{
This is an example of a thesis setup to use the reed thesis document
class (for LaTeX) and the R bookdown package, in general.
}

\Abstract{
One of the arguments against implementing ranked-choice voting (RCV) is
that RCV is harder for voters to participate in. Two of the reasons for
this are the more complicated ballot design and the extra effort that
goes into forming a ordered preference of candidates. To evaluate this
claim, we examine rates of ballot errors and undervoting (ranking fewer
than the allowed number of candidates) in some American elections
conducted with RCV. Results show that idk yet. \par

Second paragraph of abstract starts here.
}

% End of CII addition
%%
%% End Preamble
%%
%

\usepackage{amsthm}
\newtheorem{theorem}{Theorem}[chapter]
\newtheorem{lemma}{Lemma}[chapter]
\theoremstyle{definition}
\newtheorem{definition}{Definition}[chapter]
\newtheorem{corollary}{Corollary}[chapter]
\newtheorem{proposition}{Proposition}[chapter]
\theoremstyle{definition}
\newtheorem{example}{Example}[chapter]
\theoremstyle{definition}
\newtheorem{exercise}{Exercise}[chapter]
\theoremstyle{remark}
\newtheorem*{remark}{Remark}
\newtheorem*{solution}{Solution}
\begin{document}

% Everything below added by CII
  \maketitle

\frontmatter % this stuff will be roman-numbered
\pagestyle{empty} % this removes page numbers from the frontmatter
  \begin{acknowledgements}
    I want to thank a few people.
  \end{acknowledgements}
  \begin{preface}
    This is an example of a thesis setup to use the reed thesis document
    class (for LaTeX) and the R bookdown package, in general.
  \end{preface}
  \hypersetup{linkcolor=black}
  \setcounter{tocdepth}{2}
  \tableofcontents

  \listoftables

  \listoffigures
  \begin{abstract}
    One of the arguments against implementing ranked-choice voting (RCV) is
    that RCV is harder for voters to participate in. Two of the reasons for
    this are the more complicated ballot design and the extra effort that
    goes into forming a ordered preference of candidates. To evaluate this
    claim, we examine rates of ballot errors and undervoting (ranking fewer
    than the allowed number of candidates) in some American elections
    conducted with RCV. Results show that idk yet. \par
    
    Second paragraph of abstract starts here.
  \end{abstract}
  \begin{dedication}
    You can have a dedication here if you wish.
  \end{dedication}
\mainmatter % here the regular arabic numbering starts
\pagestyle{fancyplain} % turns page numbering back on

\chapter{Delete line 6 if you only have one
advisor}\label{delete-line-6-if-you-only-have-one-advisor}

Placeholder

\chapter{What is ranked choice voting?}\label{litreview}

Ranked choice voting (RCV), also known as the alternative vote (AV) or
instant-runoff voting (IRV) is an alternative voting method to the
first-past-the-post (FPTP) or ``plurality'' election system more
familiar to American voters, where the candidate with the most votes
wins. Each voter, instead of choosing their highest preference among a
set of candidates for an office, ranks some subset of the candidates in
order of preference. This system (or a close variant) is used in
Australia, Maine, and some American municipalities: San Francisco, CA;
Minneapolis, MN; and Cambridge, MA; among others.

The single-winner RCV tabulation algorithm generally proceeds as
follows:
\begin{enumerate}
\def\labelenumi{\arabic{enumi}.}
\item
  For each voter, identify their most preferred candidate that has not
  yet been eliminated. Count up these preferences by candidate.
\item
  If one candidate has a majority (50\% + 1) of the unexhausted votes,
  they are declared the winner and counting stops.
\item
  The candidate with the lowest number of votes is eliminated.
\item
  The ballots counted for that candidate are each transferred to the
  voter's next choice if one exists, or if one does not exist the ballot
  is ``exhausted'' and removed from counting for further rounds.
\item
  Return to 1.
\end{enumerate}
Most jurisdictions that use RCV have slightly different rules for edge
cases and ballot errors, but this algorithm is what distinguishes RCV
from other ranked voting systems (e.g.
\href{https://www.electoral-reform.org.uk/voting-systems/types-of-voting-system/borda-count/}{Borda},
\href{http://web.math.princeton.edu/math_alive/Voting/Lab1/Condorcet.html}{Condorcet},
\href{https://www.uk-engage.org/2013/09/electoral-systems-whats-the-difference-between-contingent-voting-and-alternative-voting-systems/}{Contingent},
etc.). A close variant of RCV is the single transferrable vote (STV)
method\footnote{More accurately, RCV is the single-winner implementation
  of the STV algorithm.}, which can be used to elect multiple
candidates, i.e.~for a school board, instead of just one. In the US,
this is used in Cambridge, MA and Minneapolis, MN to elect multi-member
offices.

\section{Frequently Used Terms}\label{frequently-used-terms}

Below are some definitions for frequently used terms later on. These are
not all ubiquitous (for example, ``undervote'' has another meaning in
most voting research), but we define them here for clarity later on.
\begin{itemize}
\item
  \emph{Overvote}: when a voter ranks multiple candidates in the same
  slot. This slot is typically thrown out entirely in counting, because
  it's often not possible to determine which candidate was preferred.
\item
  \emph{Undervote}: when a voter does not rank candidates in all of the
  slots available to them. This is different than other definitions of
  ``undervote'', which refer to a voter participating in one election on
  a ballot but not another one. This is not a problem in counting, and
  is explicitly allowed in the laws of most jurisdictions. A plurality
  election analog would be voting in high-profile races
  (e.g.~presidential), but not down-ticket decisions (e.g.~local water
  board).
\item
  \emph{Skipped vote}: when a voter ranks no candidate at slot \(x\),
  but ranks a candidate at slot \(y > x\). This is typically not a
  problem in counting, but different jurisdictions have different rules
  about whether a voter's ballot is exhausted at this point or continues
  on to their next ranked choice. Plurality voting has no analog to
  this, because each race only has one ``ranking'' (first!).
\item
  \emph{Duplicated vote}: when a voter ranks the same candidate for
  distinct slots \(x\) and \(y\). This is typically not a problem for
  counting, and the first ranking for the candidate is used. Similar to
  a skipped vote, plurality voting has no analog to this.
\item
  \emph{Ballot exhaustion}: as ballot counting progresses, some ballots
  will become ``exhausted'' when all the candidates selected are
  eliminated. Suppose the final count in an election is between
  candidates B and D, and a voter ranked candidates C-A-E. Their ballot
  would not be counted in this final round, as they expressed no
  preference for either candidate B or D. An analogous situation in a
  plurality election might be voting in the general election but not a
  runoff, that is only having a say in part of the election.
\end{itemize}
Over-, skipped, and duplicated votes are really only interpretable as
``ballot mistakes'': for example, even if a voter truly prefers two
candidates equally, the ballot instructions (should) make it clear that
ranking them at the same slot is not allowed.

\section{Claims about RCV}\label{claims-about-rcv}

There are plenty of arguments both for and against implementing RCV in
place of plurality in different jurisdictions (see the literature
review), but here we'll focus on evaluating one major argument against
it - RCV is harder for voters to participate in than a plurality system.
There are two major reasons cited for this:
\begin{itemize}
\item
  The physical design of an RCV ballot is usually more complicated than
  a plurality ballot, because there has to be a system to encode a more
  full preference among the candidates than just selecting one candidate
\item
  The process of forming a multi-candidate preference inherently takes
  more mental energy than just choosing a favorite candidate
\end{itemize}
The first facet of this argument should be reflected in ballot errors
made by voters. Compared to plurality voting, we expect more errors in
an RCV ballot just because the ballot is more complicated. There are
also more potentials for error in the RCV system generally. The only
``errors'' in a plurality ballot are incompletely marking a candidate
(think incorrect Scantron bubbling, or hanging chads) or overvoting,
both of which are potential pitfalls for a ranked choice ballot as well.
On top of these, there are the potential errors of duplicated and
skipped votes unique to ranked ballots\footnote{These types of errors
  are not uniform, and some jurisdictions are more forgiving than others
  about rules for counting these errors. While it may be apparent that a
  voter who listed the same candidate 3 times (A-A-A) prefers that
  candidate, a candidate ranking of A-B-A is harder to extract a clear
  preference from. Skipped votes are where we see the most variance in
  jurisdiction counting rules: if a voter marks the ballot A-\_\_-B,
  skipping the second slot, some jurisdictions will ignore the skip and
  treat B as the voter's second choice, while others will stop counting
  after A is eliminated (ignoring their vote for B), and others yet will
  throw out the ballot entirely.}.

The second facet should be reflected in incomplete ballots filled out by
voters. Given that they understand how to encode their preferences on
the ballot, there is still the non-trivial task of forming such a
preference. Structurally, some of the factors that should affect this
incompleteness are:
\begin{itemize}
\item
  The number of candidates running for a position
\item
  The number of candidates voters can rank
\item
  The number of seats elected in a given race
\end{itemize}
This first variable is at the election level (different for every
election), the second is at the jurisdiction level, and the third is a
mix of both. For a clear example of these differences, consider a
\href{https://sfelections.org/results/20161108/data/20161206/d3/20161206_d3.pdf}{2016
San Francisco Board of Supervisors race (District 3)} versus a
\href{https://www.cambridgema.gov/election2017/Council\%20Order\%20Round.htm}{2017
Cambridge City Council race}.
\begin{longtable}[]{@{}lll@{}}
\toprule
Factor & San Francisco 2016 & Cambridge 2017\tabularnewline
\midrule
\endhead
Candidates running & 2 & 27\tabularnewline
Candidates rankable & 2 (Generally, up to 3) & 27 (Generally,
all)\tabularnewline
Seats elected & 1 & 9\tabularnewline
\bottomrule
\end{longtable}
\section{History of RCV in the US (SF in
particular)}\label{history-of-rcv-in-the-us-sf-in-particular}

In the United States, there have been two major periods of RCV
implementation in various jurisdictions. Between 1915 and 1950, 24
American cities chose to institute RCV as a form of local election. By
1965, however, all of these except for Cambridge, MA had eliminated the
policy change. Then, in the 2000s, there was a resurgence of uptake in a
different set of American cities\footnote{Mostly in the American West:
  there are 9 cities west of the Mississippi River
  \href{https://www.fairvote.org/where_is_ranked_choice_voting_used}{currently
  using RCV} and only 4 east of it}, including Minneapolis and a handful
in the San Francisco Bay Area. While Cambridge has consistently used the
multi-winner (STV) method to elect City Council and School Board seats,
the modern resurgence of RCV almost universally deals with single-winner
elections. Research argues that RCV appears in jurisdictions where there
is strong multi-party support for the reform - the RCV method itself
gives individual parties less power in the election process, so powerful
single parties usually don't have reason to support it.

\section{Why, or why not, implement
RCV?}\label{why-or-why-not-implement-rcv}

There are plenty of arguments on both sides of implementing RCV in
jurisdictions that consider it.

\subsection{Pros}\label{pros}

\subsubsection{No secondary elections}\label{no-secondary-elections}

There are two major types of ``secondary elections'' used in American
voting: primary elections and runoff elections. Primaries are used by
political parties to select their nominee for a general election, so the
voters of any one party aren't split between different candidates.
Runoffs are most often used when no candidate in the general election
surpasses 50\% of the vote total. Typically the top two candidates from
the general election\footnote{Or primary election - Seven Southern
  states require primary winners to obtain 50\% of the vote to get on
  the general election ballot, and some other states have a requirement
  of 40\%. (WaPo article)} advance to a later runoff. These secondary
elections face two main challenges: low turnout and high cost.

Secondary elections as a whole face low turnout (Wright, 1989; Ranney,
1972). Reasons: Research shows that people don't actually like voting
that much - the more frequently elections are held, the lower turnout
will be for all of them generally (Boyd, 1986). Secondary elections
increase the number of elections in a period, so this is one possible
reason why they generally have low turnout. Further research indicates
that holding elections concurrently with a presidential election
``increase{[}s{]} the likelihood that citizens will vote'' (Boyd, 1986).
This is seen in off-year Congressinoal elections, where turnout drops
from presidential years. Typically general elections are held
concurrently with presidential elections (second Tuesday in November,
super high media coverage, lots of voter outreach, yadda yadda), so
secondary elections cannot be held at the same time as a presidential
election and they should thus suffer in turnout. This low turnout has
consequences for representation in the system. The same research
(Ranney, 1972) finds that while primary voters are not ideologically
unrepresentative of general election voters, they are both
demographically unrepresentative and unrepresentative on some major
issues. Traditional knowledge holds that primary voters are more
committed partisans than general election voters, leading the eventual
candidates in a general election to be polarized away from the
``center'' of political ideas (double check this but I'm pretty sure the
cite is Hill's \emph{Instant Runoff Voting}).

The higher costs associated with secondary elections are a little more
intuitive than turnout issues - it takes money to hold elections.
Pollworkers have to be paid, facilities have to be reserved, and
candidates have to do more campaigning. A 2011 City Council runoff in
Plano, TX cost the city an extra \$73,000 (Plano Star Courier, 2011). A
2012 Alabama runoff for multiple seats cost the state about \$3 million.
Since RCV eliminates the need for primary and runoff elections while
still ensuring majority rule (which is the main reason for these
elections), it should avoid the problems of lower turnout and higher
costs associated with secondary elections\footnote{Or at least some of
  it - see below for negative impacts on turnout and cost from RCV}.

\subsubsection{Ensures majority rule}\label{ensures-majority-rule}

In jurisdictions without rules for 50\% minimums, a common phenomenon is
a candidate winning an election with less than 50\% of the vote (a
plurality, rather than a majority). The major conceptual issue with this
is that more people preferred a candidate other than the one who was
elected\footnote{The `ideal' for electoral systems is the Condorcet
  condition: the candidate elected should beat all other candidates in
  one-on-one contests.}. RCV requires that a winning candidate receive
at least 50\% of the votes remaining\footnote{See below for issues with
  this `remaining' concept.}, ensuring that a majority of voters prefer
the elected candidate to other candidates.

This means that RCV does not always agree with the plurality method on
choosing a winner. In the Senate race in Maine between Bruce Poliquin
and Jared Golden, while Poliquin was on top at the end of the first
round of counting, neither had 50\% and Golden took the lead (and the
election) after other candidates were eliminated and their votes
transferred to voters' second choices (Portland Press-Herald).

This is particularly important in jurisdictions (like Maine) with strong
third-party support and more than two viable candidates. Former Maine
Governor Paul LePage, a Republican, won his first election in 2010 with
38.1\% of the vote, compared to Independent Eliot Cutler's
36.7\%\footnote{a margin of about 7,500 votes. Democrat Libby Mitchell
  received 19\%.}.

\subsubsection{Effect on spoiler candidates and third
parties}\label{effect-on-spoiler-candidates-and-third-parties}

The ``spoiler effect'' is when a third party candidate draws votes away
from the ideologically closest major party candidate, thus contributing
to the election of the other major party candidate. The most recent
large-scale accusation of this was in the 2000 election. Green Party
candidate Ralph Nader drew about 3\% of the national vote, more than the
margin of victory for George W. Bush over Al Gore\footnote{Admittedly,
  the margins are less clear-cut than this at the state level, where the
  margins actually matter for the Electoral College.}. In the especially
consequential state of Florida, Nader took 1.6\% of the vote: almost 200
times greater than the margin between the two major party candidates of
less than .01 percentage points\footnote{Not to point fingers at Nader
  alone in this case - while he was the most popular third party
  candidate by far, all 8 official third-party candidates received more
  votes than the major-candidate margin of only 537 votes} (source from
FEC). Many believed that Nader, generally seen as more liberal than the
Democrat Gore, drew votes from the Democratic base that would have
helped Gore win the election otherwise. While research into third-party
voters casts some doubt on this theory's applicability in 2000 (Herron,
Lewis)\footnote{In short - while Nader's Florida voters potentially
  would have broken enough for Gore to put him over the top, this was
  more a factor of the extreme closeness between the two major
  candidates than anything that Nader aided in particular.}, public
opinion still rests on the idea that Nader cost Gore the
presidency\footnote{One of the sections of Nader's Wikipedia page is
  entitled ``Spoiler controversy'' in regards to this election.}.

This also helps third-party candidates get elected, because voters can
ignore this facet of strategic voting\footnote{Or any strategic voting -
  while not impossible, it's infeasible to vote strategically under RCV
  (source from the computational paper).} and select their truly
preferred candidate. Due to the spoiler problem, voters who want to vote
for a third-party candidate are incentivized to vote for the major
candidate they prefer to avoid the less preferred candidate from being
elected - a situation of ``the lesser of two evils'', so to speak.

\subsubsection{Disincentivizes negative
campaigning}\label{disincentivizes-negative-campaigning}

Ranked choice voting should incentivize candidates to avoid negative
campaigning. In a plurality election, since candidates don't care about
voters who are committed to their competitors, a well-thought out
negative campaign will only ostracize voters who were never going to
support another candidate in the first place, and perhaps bring more
swing voters to their side. Under RCV, however, alienating another
candidate's voters could backfire in the event that candidate is
eliminated and these voters decide to support your opponent in the next
round, causing your defeat . An interesting real-life case of this is in
the 2018 San Francisco mayoral election . There were three frontrunners
heading into election day, all incumbent members of the city's Board of
Supervisors: London Breed, Jane Kim, and Mark Leno. As polls showed
Breed ahead about a month before the election, Kim and Leno held a joint
press conference to endorse the other as voters' second choices. By
drawing second-choice votes from the other candidate, the remaining
candidate hoped to overcome the gap between them and Breed. In the
actual election, the standing when it came time to proceed to the final
round of counting was 102,767 for Breed, 68,707 for Leno, and 66,043 for
Kim. While a significant proportion of Kim's voters transferred to Leno
after her elimination, in the final round Breed surpassed Leno by about
2,000 votes.

Though it's outside the scope of this research to tell if this
cross-endorsement was effective\footnote{Other confounding factors
  counld exist: maybe Kim and Leno had similar enough positions that
  this scenario would have happened without the endorsement, maybe this
  number is only significant because in the final rounds there were only
  2 candidates for second choice votes to flow to, etc.}, there is some
evidence in favor of this theory. Leno received almost 70\% of the votes
previously counted for Kim compared to Breed's 20\%, bringing Breed's
final margin of victory down to only 1 percentage point. In previous
rounds of the election, no single candidate ever received more than 35\%
of the transferred votes from an eliminated candidate\footnote{Except
  for round 2, where all 3 votes for the same write-in candidate
  transferred to Breed.}, so this is at least an unusual observation.

\subsubsection{Minority Candidate
Election}\label{minority-candidate-election}

Imagine a city that consistently had 45\% of voters pick Party A and
55\% pick Party B. The minority in party A would never be able to elect
a candidate under plurality voting because they would lose the one
election every year - the representation of the city would be 100\%
Party B. If the district elected two seats under STV, however, there
would be one candidate from each party - the city's representation would
be 50\% Party A and 50\% Party B, which is much representative of the
population's actual views\footnote{The study of gerrymandering uses this
  type of analysis, involving a metric called the ``efficiency gap''.}.
The benefit of having multi-member districts is the ability for minority
groups in the city to be represented.

While the example above used political parties, consider the application
of multi-member districts to racial or ethnic groups. In Cambridge, with
an African-American population of roughly 5-10\% of the city over time,
there has consistently been a black member elected to the City Council.
Since there are 9 seats on the City Council, the threshold for electing
candidates is only 10\% (instead of 50\%), so minority groups are much
more likely to elect their preferred candidates to office.

\subsubsection{Turnout Improvements}\label{turnout-improvements}

All of these pros boost turnout, because people are more trusting that
their government actually represents them.

\subsection{Cons}\label{cons}

Doesn't \emph{really} ensure a majority, just a majority of final voters

Less extant infrastructure - sometimes handcounted, non-instant results

Legal challenges (Maine constitution)

Voter education problems - ballot errors, decreased turnout

Less intuitive rules (perhaps fold into the previous)

Lack of true adoption by voters, only listing first choice (perhaps also
fold into previous)

\section{Research into SF?}\label{research-into-sf}

\chapter{Methods and Structure}\label{methods}

Placeholder

\subsection{Data Structure and Source}\label{data-structure-and-source}

\chapter{This chunk ensures that the thesisdown package
is}\label{this-chunk-ensures-that-the-thesisdown-package-is}

Placeholder

\section{Tables}\label{tables}

\section{Figures}\label{figures}

\section{Footnotes and Endnotes}\label{footnotes-and-endnotes}

\section{Bibliographies}\label{bibliographies}

\section{Anything else?}\label{anything-else}

\chapter*{Conclusion}\label{conclusion}
\addcontentsline{toc}{chapter}{Conclusion}

If we don't want Conclusion to have a chapter number next to it, we can
add the \texttt{\{-\}} attribute.

\textbf{More info}

And here's some other random info: the first paragraph after a chapter
title or section head \emph{shouldn't be} indented, because indents are
to tell the reader that you're starting a new paragraph. Since that's
obvious after a chapter or section title, proper typesetting doesn't add
an indent there.

\appendix

\chapter{The First Appendix}\label{the-first-appendix}

This first appendix includes all of the R chunks of code that were
hidden throughout the document (using the \texttt{include\ =\ FALSE}
chunk tag) to help with readibility and/or setup.

\textbf{In the main Rmd file}

\textbf{In Chapter \ref{ref-labels}:}

\chapter{The Second Appendix, for
Fun}\label{the-second-appendix-for-fun}

\chapter*{References}\label{references}
\addcontentsline{toc}{chapter}{References}

Placeholder


% Index?

\end{document}
